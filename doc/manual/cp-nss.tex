\chapter{Named Saved Systems}
\label{chap:cp-nss}
\cbstart
A Named Saved System (NSS) is saved storage image that can be IPLed under a
given name.  The image can be reloaded and execution resumed.

NSS improve performance by allowing sharing of static pages between guests.
Additionally, these static pages, if unmodified by the guest, do not have to
be paged out.  Instead, they may be simply discarded and later reloaded from
the NSS.

The contents of the storage image may be thought of as the contents of
storage at any time during a systems execution.
%
It is possible to take a running system, quiesce all I/O and save the
storage contents.  Since the contents of the registers and I/O device states
are not explicitly stored, the system must take care to store any and all
state it wishes to keep.  A simple way to store all the processor state is
to issue a store status operation.

\section{Named Saved System File Format}
Each NSS is saved in a single file using the ELF file format.  (See
section~\ref{sec:elf} for ELF file format details.)  The ELF header class
CLASS32 indicates a \mbox{31-bit} system, while CLASS64 indicates a
\mbox{64-bit} system.

The guest storage is saved as one or more segments. Each segment's type is
LOAD.  The guest registers and other system state is not stored.

Outcome of loading a NSS with overlapping segments is unpredictable.

HVF has two variants for NSS formats.  For both formats, the NSS IPL process
loads the storage image from the NSS file.  The only difference between the
two formats are the source of the new PSW.

\subsection{Load-Format}
The PSW is cleared.  If the saved system is a \mbox{31-bit} image, bit~12 of
the PSW is set.  The PSW address is set to the ELF header's entry address.

The Load-Format is denoted by ELF header type of EXEC.

\subsection{Restart-Format}
\note{HVF currently does not support this format.}

The PSW is loaded storage location denoting the new restart PSW.  The
address from which the restart PSW is loaded depends on the architecture
mode.

The Restart-Format is denoted by ELF header type of CORE.
\cbend
