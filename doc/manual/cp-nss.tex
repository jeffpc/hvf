\chapter{Named Saved Systems}
\label{chap:cp-nss}
\cbstart
\note{HVF does not currently support NSS IPL and therefore the following is
	the description of the design to be implemented.}

A Named Saved System (NSS) is saved storage image that can be IPLed under a
given name.  The image can be reloaded and execution resumed.

The contents of the storage image may be thought of as the contents of
storage right after the conclusion of the IPL channel program for a CCW-Type
IPL.

Of course it is possible to take a running system place an appropriate PSW
at address 0.  Since the contents of the registers and I/O device states are
not stored, the system must take care to store any and all state it wishes
to keep.  A simple way to store all the processor state is to issue a store
status operation.

NSS improve performance by allowing sharing of static pages between guests.
Additionally, these static pages, if unmodified by the guest, do not have to
be paged out.  Instead, they may be simply discarded and later reloaded from
the NSS.

\section{Named Saved System File Format}
Each NSS is saved in a single file using the ELF file format.  (See
section~\ref{sec:elf} for ELF file format details.)

The guest storage is saved as one or more sections. Each section's type may
be either PROGBITS or NOBITS.  Any PROGBITS sections are loaded verbatim at
the specified address.  Any NOBITS sections cause the specified storage
range to be cleared.

Outcome of loading a NSS with overlapping sections is unpredictable.
\cbend
