\chapter{CP Programming Interfaces}
\section{ABI}
HVF uses the same ABI (s390x) as Linux on z.  The following sections attempt
to introduce portions of it.  For more information see the relevant
literature.

\subsection{Register Assignments}
The ABI defines these register assignments.

\begin{center}
\begin{tabular}{|l|l|l|}
\hline
Reg   &   Use                                        &Saved? \\
\hline\hline
R0    &   Used by syscalls/assembly                  &Call-clobbered \\
R1    &   Used by syscalls/assembly                  &Call-clobbered \\
R2    &   Argument 0 / return value 0                &Call-clobbered \\
R3    &   Argument 1 / return value 1 (if long long) &Call-clobbered \\
R4    &   Argument 2                                 &Call-clobbered \\
R5    &   Argument 3                                 &Call-clobbered \\
R6    &   Argument 4                                 &Saved \\
R7    &   Pointer-to arguments 5 to ...              &Saved \\
R8    &   This \& that                               &Saved \\
R9    &   This \& that                               &Saved \\
R10   &   Static-chain ( if nested function )        &Saved \\
R11   &   Frame-pointer ( if function used alloca )  &Saved \\
R12   &   Got-pointer                                &Saved \\
R13   &   Base-pointer                               &Saved \\
R14   &   Return-address                             &Saved \\
R15   &   Stack-pointer                              &Saved \\
\hline
\end{tabular}
\end{center}

Note that HVF using this ABI internally places no restrictions on the guest
operating systems in any way.

\section{ELF Binary File Format}
FIXME: describe ELF
\section{Remote Procedure Call Interface}
FIXME: describe the rpc
