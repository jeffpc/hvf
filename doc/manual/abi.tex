\chapter{ABI}
\label{chap:abi}
HVF uses the same ABI (s390x) as Linux on z.  The following sections attempt
to introduce portions of it.  For more information see the relevant
literature.

Note that HVF using this ABI internally places no restrictions on the guest
operating systems in any way.

\section{Register Assignments}
\label{sec:registers}
The ABI defines these register assignments.

\cbstart
All HVF code written in C follows the below assignment.

Any assembly helpers interfacing with C code must at the very least follow
this convention at the entry and exit points, however internally they may
use any register assignment.

All other assembly code (the assembly portion of the IPL loader, the guest
IPL helper, NSS, etc.) is free to follow any convention.  In this case, the
code clearly documents any register convention.
\cbend

\begin{center}
\begin{tabular}{|l|l|l|}
\hline
Reg   &   Use                                        &Saved? \\
\hline\hline
R0    &   Used by syscalls/assembly                  &Call-clobbered \\
R1    &   Used by syscalls/assembly                  &Call-clobbered \\
R2    &   Argument 0 / return value 0                &Call-clobbered \\
R3    &   Argument 1 / return value 1 (if long long) &Call-clobbered \\
R4    &   Argument 2                                 &Call-clobbered \\
R5    &   Argument 3                                 &Call-clobbered \\
R6    &   Argument 4                                 &Saved \\
R7    &   Pointer-to arguments 5 to ...              &Saved \\
R8    &   This \& that                               &Saved \\
R9    &   This \& that                               &Saved \\
R10   &   Static-chain ( if nested function )        &Saved \\
R11   &   Frame-pointer ( if function used alloca )  &Saved \\
R12   &   Got-pointer \cbstart / local variable\cbend &Saved \\
R13   &   Base-pointer \cbstart (aka. literal pool pointer)\cbend &Saved \\
R14   &   Return-address                             &Saved \\ % verify!
R15   &   Stack-pointer                              &Saved \\
\hline
\cbstart
F0    &   Argument 0 / return value 0                &Call-clobbered \\
F1    &   General purpose                            &Call-clobbered \\
F2    &   Argument 1                                 &Call-clobbered \\
F3    &   General purpose                            &Call-clobbered \\
F4    &   argument 2 (z/Architecture only)           &Saved \\
F5    &   General purpose                            &Call-clobbered \\
F6    &   argument 3 (z/Architecture only)           &Saved \\
F7    &   General purpose                            &Call-clobbered \\
F8    &   General purpose                            &Call-clobbered \\
F9    &   General purpose                            &Call-clobbered \\
F10   &   General purpose                            &Call-clobbered \\
F11   &   General purpose                            &Call-clobbered \\
F12   &   General purpose                            &Call-clobbered \\
F13   &   General purpose                            &Call-clobbered \\
F14   &   General purpose                            &Call-clobbered \\
F15   &   General purpose                            &Call-clobbered \cbend \\
\hline
\end{tabular}
\end{center}

\cbstart
\section{Stack Frame Layout}
\label{sec:stackframe}
\begin{center}
\begin{tabular}{|l|l|l|}
\hline
z/Arch (s390x) & s/390 (s390) & Use \\
\hline\hline
0            & 0        & back chain (NULL signifies end of back chain)\\
8            & 4        & end of stack (not used in Linux s390/s390x)\\
16           & 8        & glue (not used in Linux s390/s390x)\\
24           & 12       & glue (not used in Linux s390/s390x)\\
32           & 16       & scratch area\\
40           & 20       & scratch area\\
48           & 24       & saved r6 of caller function\\
56           & 28       & saved r7 of caller function\\
64           & 32       & saved r8 of caller function\\
72           & 36       & saved r9 of caller function\\
80           & 40       & saved r10 of caller function\\
88           & 44       & saved r11 of caller function\\
96           & 48       & saved r12 of caller function\\
104          & 52       & saved r13 of caller function\\
112          & 56       & saved r14 of caller function\\
120          & 60       & saved r15 of caller function\\
128          & 64       & saved f4 of caller function\\
132          & 72       & saved f6 of caller function\\
             & 80       & undefined\\
160          & 96       & outgoing args passed from caller to callee\\
160+x        & 96+x     & possible stack alignment (8 bytes desirable)\\
160+x+y      & 96+x+y   & alloca space of caller (if used)\\
160+x+y+z    & 96+x+y+z & automatics of caller (if used)\\
\hline
\end{tabular}
\end{center}
\cbend

\section{ELF Binary File Format}
\label{sec:elf}
This section introduces the ELF binary file format for the \cbstart s390 and
s390x ABIs\cbend.  For more detailed description, see the appropriate
standards.  \cbstart All \mbox{31-bit} versions of the structures are named
with the \mbox{Elf32\_} prefix, while the \mbox{64-bit} versions use the
\mbox{Elf64\_} prefix.\cbend

\cbstart
Each ELF file is made of sections.  Each section defines a contiguous range
of bytes.  Each section has a header that specifies how to handle the
section.

Not all sections need to be loaded to allow program execution.  For example,
the symbol table is stored as a section with sh\_type equal to SHT\_SYMTAB
(for definition of sh\_type and SHT\_SYMTAB, see below).

Not all sections may contain data in the ELF file.  For example, the .bss
section (in C applications for UNIX) gets initialized to binary zeros.  The
ELF file simply contains the header for the section, along with the
SHT\_NOBITS type.  This instructs the loader to allocate the necessary
storage, and clear it.  (For definition of SHT\_NOBITS, see below).

Each ELF file begins with a header.  The header contains several pointers to
other structures in the file, such as the section headers.

The ELF header points to program headers.  A program header describes a
segment (not to be confused with a 1~MB area of storage) that spans one or
more contiguously stored sections.

Program headers are meant to be used during program loading, while sections
are supposed to be used during linking.
\cbend

\cbstart
\subsection{ELF Header}
\cbend

\cbdelete
\begin{lstlisting}[language=C]
typedef struct {
        u8       e_ident[16];
        u16      e_type;
        u16      e_machine;
        u32      e_version;
        u64      e_entry;
        u64      e_phoff;
        u64      e_shoff;
        u32      e_flags;
        u16      e_ehsize;
        u16      e_phentsize;
        u16      e_phnum;
        u16      e_shentsize;
        u16      e_shnum;
        u16      e_shstrndx;
} Elf64_Ehdr;
\end{lstlisting}

\cbstart
\begin{lstlisting}[language=C]
typedef struct {
        u8       e_ident[16];
        u16      e_type;
        u16      e_machine;
        u32      e_version;
        u32      e_entry;
        u32      e_phoff;
        u32      e_shoff;
        u32      e_flags;
        u16      e_ehsize;
        u16      e_phentsize;
        u16      e_phnum;
        u16      e_shentsize;
        u16      e_shnum;
        u16      e_shstrndx;
} Elf32_Ehdr;
\end{lstlisting}
\cbend

\begin{description}
\item[e\_ident] contains several bytes of information:
	
	Bytes 0-3 contain the magic ASCII string ``\textbackslash x7FELF'', which is
	equal to 7F454C46 hex.

	Byte 4 contains the ELF class which should equal 1 \cbstart
	(ELFCLASS32) \cbend for 31-bit files, and 2 \cbstart (ELFCLASS64)
	\cbend for 64-bit files.

	Byte 5 contains the byte order information.  A value of 1 \cbstart
	(ELFDATA2LSB) \cbend indicates little endian format, while value of
	2 \cbstart (ELFDATA2MSB) \cbend indicates a big endian format.  In
	HVF this must always be equal to 2.

	Byte 6 contains the ELF file format version.  Currently, only one
	version is defined --- 1.

	Bytes 7-15 contain other information, however HVF ignores these
	fields.
	% FIXME: fill this in later

\item[e\_type] specifies the type of the payload.

	\cbstart
	Valid values are: 0 (ET\_NONE), 1 (ET\_REL), 2 (ET\_EXEC), 3
	(ET\_DYN), and 4 (ET\_CORE)
	\cbend

	Currently, HVF expects only executables, and therefore this field
	should be set to 2 \cbstart (ET\_EXEC) \cbend.

\item[e\_machine] identifies the target architecture.  It should always
	equal 16 hex identifying it as ESA/390 or z/Architecture \cbstart
	(EM\_S390) \cbend.

\item[e\_version] contains the ELF header version number.  It should always
	equal to 1.

\item[e\_entry] is the address of the entry point.  That is the location of
	the instruction to transfer control to once loading has completed.

\item[e\_phoff] contains the offset into the file to the first program
	header.  HVF ignores this field.

\item[e\_shoff] contains the offset into the file to the first section
	header.

\item[e\_flags] contains processor specific flags.  HVF ignores this field.

\item[e\_ehsize] is the ELF header size.  HVF ignores this field.

\item[e\_phentsize] equals the program header entry size.  HVF ignores this
	field.

\item[e\_phnum] is the number of program headers.  HVF ignores this field.

\item[e\_shentsize] equals the section header entry size.

\item[e\_shnum] is the number of section header.

\item[e\_shstrndx] contains the section name string table index.  HVF
	ignores this field.
\end{description}

\cbstart
\subsection{Section Header}
Each section header is as follows:

\begin{lstlisting}[language=C]
typedef struct {
        u32      sh_name;
        u32      sh_type;
        u64      sh_flags;
        u64      sh_addr;
        u64      sh_offset;
        u64      sh_size;
        u32      sh_link;
        u32      sh_info;
        u64      sh_addralign;
        u64      sh_entsize;
} Elf64_Shdr;
\end{lstlisting}

\cbstart
\begin{lstlisting}[language=C]
typedef struct {
        u32      sh_name;
        u32      sh_type;
        u32      sh_flags;
        u32      sh_addr;
        u32      sh_offset;
        u32      sh_size;
        u32      sh_link;
        u32      sh_info;
        u32      sh_addralign;
        u32      sh_entsize;
} Elf32_Shdr;
\end{lstlisting}
\cbend

\begin{description}
\item[sh\_name] is the name of the section.  HVF ignores this field.

\item[sh\_type] defines the type of the section.

	Valid types include SHT\_NULL (00), SHT\_PROGBITS (01), SHT\_SYMTAB
	(02), SHT\_STRTAB (03), SHT\_RELA (04), SHT\_HASH (05), SHT\_DYNAMIC
	(06), SHT\_NOTE (07), SHT\_NOBITS (08), SHT\_REL (09), SHT\_SHLIB
	(0A), and SHT\_DYNSYM (0B).

	HVF handles only sections SHT\_PROGBITS with non-zero sh\_addr (see
	below), and SHT\_NOBITS.

\item[sh\_flags] defines the section attributes.

\item[sh\_addr] stores the target in-storage address for this section.

\item[sh\_offset] stores the offset into the file where section data begins.

\item[sh\_size] stores the length of the section.

\item[sh\_link] contains links to other sections.  HVF ignores this field.

\item[sh\_info] contains miscellaneous information.  HVF ignores this field.

\item[sh\_addralign] specifies the address alignment boundary.  This field
	is useful for relocatable sections.  Since HVF does not support
	those, it ignores this field.

\item[sh\_entsize] defines the size of entries if the section contains a
	table.  Since HVF does not parse any sections with tables, it
	ignores this field.

\end{description}

\subsection{Program Header}
Each program header is as follows:

\begin{lstlisting}[language=C]
typedef struct {
        u32      p_type;
        u32      p_flags;
        u64      p_offset;
        u64      p_vaddr;
        u64      p_paddr;
        u64      p_filesz;
        u64      p_memsz;
        u64      p_align;
} Elf64_Phdr;
\end{lstlisting}

\begin{lstlisting}[language=C]
typedef struct {
        u32      p_type;
        u32      p_offset;
        u32      p_vaddr;
        u32      p_paddr;
        u32      p_filesz;
        u32      p_memsz;
        u32      p_flags;
        u32      p_align;
} Elf32_Phdr;
\end{lstlisting}

\begin{description}
\item[p\_type] defines the type of the segment.

	Valid types include PT\_NULL (00), PT\_LOAD (01), PT\_DYNAMIC (02),
	PT\_INTERP (03), PT\_NOTE (04), PT\_SHLIB (05), PT\_PHDR (06),
	PT\_TLS (07).

	HVF handles only segments PT\_LOAD.

\item[p\_flags] defines the segment attributes.

\item[p\_offset] stores the offset into the file where segment data begins.

\item[p\_vaddr] stores the in-storage address for this segment.

\item[p\_paddr] is not used in the s390 and s390x ABIs.

\item[p\_filesz] stores the length of the segment stored in the file.

\item[p\_memsz] stores the length of the segment once loaded into storage.
	p\_memsz may be larger than p\_filesz, in which case the remaining
	bytes of storage are cleared.

\item[p\_align] indicates the in-storage alignment of this segment.
\end{description}
\cbend
