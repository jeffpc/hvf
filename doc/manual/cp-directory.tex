\chapter{CP Directory}
\label{chap:directory}
The directory contains a list of all valid users on the system.  Currently,
any changes to the directory require that the nucleus be rebuilt, and the
system re-IPLed.

\section{Directory Syntax}
The directory is a simple text file with a declarative syntax used to
describe all the virtual machines.

All empty lines and lines with only white space are ignored.

\cbstart
Each virtual machine definition must begin with a USER statement specifying
the userid and the authorization level.  All statements following the USER
statement are applied to this userid definition until another USER statement
is encountered.

\begin{syntdiag}
\tok{USER} <userid> <auth>
\end{syntdiag}
\cbend

A \cbstart MACHINE\cbend statement specifies the architectural level and
number of virtual processors.  Currently, HVF ignores this statement and
always instantiates a single z/Architecture virtual processor.

\cbstart
\begin{syntdiag}
\tok{MACHINE} <type> <numcpus>
\end{syntdiag}
\cbend

The STORAGE statement may appear only once within a virtual machine's
definition.  It specifies the amount of memory allocated to the guest.  The
size is scaled by the optional scale --- K (\mbox{1,024}), M
(\mbox{1,048,576}), or G (\mbox{1,073,741,824}).

Note that the size must be a multiple of 64kB and that the directory parser
does not check for this.

\begin{syntdiag}
\tok{STORAGE} <size>
\begin{stack}
	\\
	\tok{K} \\
	\tok{M} \\
	\tok{G}
\end{stack}
\end{syntdiag}

\cbstart
The CONSOLE statement defines a system console.  There can be only one
CONSOLE statement per virtual machine definition of type 3125.  Currently,
HVF does not check for this.

\begin{syntdiag}
\tok{CONSOLE} <vdev> <type>
\end{syntdiag}

The SPOOL statement may appear arbitrary number of times per virtual machine
definision.  Each occurence defines a virtual device with the virtual device
number vdev handled by HVF's spooling facility.

The expected type for a READER is 3505.  The expected type for a PUNCH is
3525.  The expected type for PRINT is 1403.

\begin{syntdiag}
\tok{SPOOL} <vdev> <type>
\begin{stack}
	\tok{READER}\\
	\tok{PUNCH} \\
	\tok{PRINT}
\end{stack}
\end{syntdiag}

The MDISK statement defines a minidisk residing on rdev (a DASD of type
type) starting at cylinder off and being len cylinders long.

\begin{syntdiag}
\tok{MDISK} <vdev> <type> <off> <len> <rdev>
\end{syntdiag}

The DEDICATE statement defines a dedicated virtual device.

\begin{syntdiag}
\tok{DEDICATE} <vdev> <rdev>
\end{syntdiag}
\cbend

\section{Example Directory}
Figure~\ref{fig:directory-sample} is the default directory distributed with
HVF.  It defines several users with nearly identical configurations.

\begin{figure*}[htb]
\small
\lstinputlisting{../../sys/hvf.directory}
\captionfont
\caption{\capfont Example directory defining three users.}
\label{fig:directory-sample}
\end{figure*}

