\chapter{CP Directory}
\label{chap:directory}
The directory contains a list of all valid users on the system.  Currently,
any changes to the directory require that the nucleus be rebuilt, and the
system re-IPLed.

\section{Directory Syntax}
The directory is a simple text file with a declarative syntax used to
describe all the virtual machines.

All empty lines and lines with only white space are ignored.

Each virtual machine definition must begin with a USERID statement
specifying the userid and the authorization level.  All statements following
the USERID statement are applied to this userid definition until another
USERID statement is encountered.

\begin{syntdiag}
\tok{USERID} <userid> <auth>
\end{syntdiag}

A CPU statement defines a virtual processor.  The cpuid parameter defines
the number of the cpu in hex.  Typically, the first processor is numbered
00.  Currently, HVF ignores this statement and always instantiates a single
guest processor.

\begin{syntdiag}
\tok{CPU} <cpuid>
\end{syntdiag}

The STORAGE statement may appear only once within a virtual machine's
definition.  It specifies the amount of memory allocated to the guest.  The
size is scaled by the optional scale --- K (\mbox{1,024}), M
(\mbox{1,048,576}), or G (\mbox{1,073,741,824}).

Note that the size must be a multiple of 64kB and that the directory parser
does not check for this.

\begin{syntdiag}
\tok{STORAGE} <size>
\begin{stack}
	\\
	\tok{K} \\
	\tok{M} \\
	\tok{G}
\end{stack}
\end{syntdiag}

The DEV statement may appear arbitrary number of times.  Each occurence
defines a virtual device with the virtual device number vdev.  The device
types CONSOLE, RDR, PUN, and PRT connect the device to CP's spooling
facility which then directs any I/Os to the appropriate destination.  When
the MDISK device type is specified, the virtual device is a minidisk
residing on rdev (a DASD) starting at cylinder off and being len cylinders
long.

\begin{syntdiag}
\tok{DEV} <vdev>
\begin{stack}
	\tok{CONSOLE} \\
	\tok{RDR} \\
	\tok{PUN} \\
	\tok{PRT} \\
	\tok{MDISK} <rdev> <off> <len>
\end{stack}
\end{syntdiag}

\section{Example Directory}
Figure~\ref{fig:directory-sample} is the default directory distributed with
HVF.  It defines several users with nearly identical configurations.

\begin{figure*}[htb]
\small
\lstinputlisting{../../sys/hvf.directory}
\captionfont
\caption{\capfont Example directory defining three users.}
\label{fig:directory-sample}
\end{figure*}

